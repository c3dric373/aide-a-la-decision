\documentclass{article}
\usepackage[bottom,flushmargin,hang,multiple]{footmisc}
\usepackage[utf8]{inputenc}
\usepackage[francais]{babel}
\usepackage{fancyhdr}
\usepackage{amsmath}
\usepackage{graphicx}
\usepackage{float}
\usepackage{amssymb}
\usepackage{geometry}
\usepackage{caption}
\usepackage{subcaption}
\geometry{margin=2cm}
\usepackage{tabularx}
\usepackage{longtable}
\usepackage{hyperref}
\hypersetup{
    colorlinks=false, %set true if you want colored links
    linktoc=all,     %set to all if you want both sections and subsections linked
}
\usepackage{enumitem}
\setlist[itemize]{before={\vspace{2mm}},after={\vspace{2mm}}}
\newcommand{\cref}[1]{Figure \ref{#1}}
\title{{\Huge Projet aide à la décision}}

\author{{\Large
H4222}\\
-
\\
Félix Castillon, Roxane Debord, David Hamidovic, Corentin Laharotte,\\
Cédric Milinaire, Grazia Ribbeni, Ousmanne Touat\\
-}

\date{Février 2020}

\pagestyle{fancy}
\fancyhf{}
\lhead{TP Fouille de données}
\rfoot{Page \thepage}

\begin{document}

\maketitle
\thispagestyle{empty}

\section{Introduction}
En vue de trouver un nouveau plan hebdomadaire de production nous allons réaliser un état de l’art du plan existant en vue de l’améliorer. Pour cela, différentes démarches seront mises en œuvre.\\

Dans un premier temps nous allons concevoir une solution spécifique pour chacune des parties évoquées dans l’appel d’offre : comptabilité, fabrication, stock, commercial, personnel. Il s'agira donc d'une optimisation mono-critère.\\
Ensuite nous allons aborder le sujet de façon plus globale en effectuant une optimisation multi-critère qui prend en compte chaque solution spécifique identifiée auparavant. Il sera alors possible d'identifier le point de mire et la matrice de gain ainsi que différentes solutions possibles parmi lesquelles il y aura celle que nous allons vous proposer.\\
Dernièrement une deuxième analyse multi-critère sera effectuée à partir de 8 proposition de gestion de l'atelier selon 4 critères sélectionnés.


\section{Analyses mono-critère}
\subsection{Comptabilité}

But : Maximiser le bénéfice en tenant compte des coûts de production et d’approvisionnement\\

Démarche :\\
Analyse et modélisation du problème :\\
Nous disposons de : \\

\begin{itemize}
\item Prix de vente des produits (table 4a),
\item Prix d’achat de chaque matière première (table 4b),
\item Quantités des matières premières nécessaire pour chaque produit (table 2),
\item Quantités maximales de stockage de chaque matière première (table 3),
\item Temps d’usinage de chaque produit sur chaque machine (table 1),
\item Coût horaire de chaque machine (table 5)
\end{itemize}

Pour modéliser le problè

\subsection{Responsable des Stocks}

Le but du responsable des stocks est de minimiser le nombre de produits dans son stock, tout en garantissant l'activité de l'entreprise.

Pour résoudre ce problème, nous avons réfléchi à la manière la plus pertinente possible de minimiser le stock ( le fait de ne pas faire de stock étant une solution minimale, mais cette solution suppose que l'activité de l'entreprise soit nulle ). Nous avons donc cherché à définir ce qu'était une entreprise en activité, et nous sommes arrivés à 3 principes  :
\begin{itemize} 
\item L'entreprise produit un certain pourcentage de sa production maximale
\item L'entreprise réalise un pourcentage de son bénéfice maximal
\item L'entreprise a des machines en activité
\end{itemize} \ \\ 

Nous avons choisi le premier critère pour définir l'activité de l'entreprise. \\
Nous avons donc défini une fonction permettant de calculer le stock global de l'entreprise : le stock est calculé par la somme des produits réalisés et des matières premières utilisées. Chaque produit fabriqué prend une unité dans le stock et chaque matière utilisée pour fabriquer le produit prend aussi une unité dans le stock. Ce qui donne :
\[
   Stock(X) =  
   (
   \begin{pmatrix} 
   1 & 1 & 1 
   \end{pmatrix}
   * M2 +
   \begin{pmatrix} 
   1 & 1 & 1 & 1 & 1 & 1
   \end{pmatrix} 
   )*X
   \]
 avec X= $\begin{pmatrix} 
   x1 \\ 
   x2 \\
   x3 \\
   x4 \\
   x5 \\ 
   x6 
  \end{pmatrix}$ représentant le nombre de produit de chaque type ( x1 est le nombre de A, x2 est le nombre de B,...). \\
  Et M2 étant la matrice correspondant au tableau des quantités de matières nécessaires pour fabriquer les produits. \\
  
Donc :
\[ 
   Stock(X) =  
   (
   \begin{pmatrix} 
   1 & 1 & 1 
   \end{pmatrix}
   *
   \begin{pmatrix} 
   1 & 2 & 1 & 1 & 1 & 2 \\ 
   2 & 2 & 1 & 2 & 2 & 1 \\
   1 & 0 & 3 & 2 & 2 & 0
   \end{pmatrix} 
   +
   \begin{pmatrix} 
   1 & 1 & 1 & 1 & 1 & 1
   \end{pmatrix} 
   )*X
\]
Soit :
\[ 
   Stock(X) =  
   (
   \begin{pmatrix} 
   4 & 4 & 5 & 5 & 5 & 3 \\
   \end{pmatrix}
   +
   \begin{pmatrix} 
   1 & 1 & 1 & 1 & 1 & 1
   \end{pmatrix} 
   )*X
\]
Soit :
\[ 
   Stock(X) =  
   \begin{pmatrix} 
   5 & 5 & 6 & 6 & 6 & 4
   \end{pmatrix}
	*X
\]

Il reste maintenant à minimiser le stock tout en garantissant l'activité de l'entreprise. Pour cela, nous avons cherché le pourcentage optimal tel que le stock soit minimal et que le stock soit supérieur au pourcentage de la  production maximale de l'entreprise, soit $Stock(X)> P_{opt}*X_{Max}$.\\
Nous avons tracé le graphe du stock minimal correspondant aux critères en fonction du pourcentage de production maximal. Nous avons obtenu le graphe suivant : 

\begin{center}
\begin{figure}[H]
\includegraphics[width=1\textwidth]{img/Stock_Activite}
\caption{Stock en fonction du pourcentage d'activité}
\end{figure}
\end{center}

Nous pouvons remarquer qu'il y a 2 ruptures dans le tracé de cette fonction : 
\begin{itemize}
\item Une première rupture vers de 43\% de production maximale
\item Une deuxième rupture vers 93\% de production maximale
\end{itemize}

Pour connaître plus précisément ces positions de ruptures, nous avons tracé les courbes de tendances correspondant à chaque morceau. Nous avons obtenu les résultats suivants :

\begin{center}
\begin{figure}[H]
\includegraphics[width=1\textwidth]{img/Stock_Activite_tendance}
\caption{Stock en fonction du pourcentage d'activité avec courbes de tendance}
\end{figure}
\end{center}

Ces résultats nous permettre de déterminer les points de rupture :
\begin{itemize}
\item Premier point de rupture : intersection des 2 premières droites :\\
$16.9*P_{opt} = 21.3*P_{opt} - 200$ \\
Soit $P_{opt} = 45.5 \approx 45 $
\item Deuxième point de rupture : intersection des 2 dernières droites :\\
$41.2*P_{opt}-2043 = 21.3*P_{opt} - 200$ \\
Soit $P_{opt} = 92.6 \approx 93 $
\end{itemize}

Ces points sont importants puisque ce sont des points de rupture avant une grande augmentation des stocks. Le stock minimal est celui correspondant à une activité de 45\%, soit 762.2 unités. Cependant, il est difficile de demander à une entreprise de ne produire que 50\% de sa production maximale, c'est pourquoi l'entreprise pourrait choisir de produire 93\% de sa capacité maximale et avoir un stock de 1799.6 unités. \\

\section{Analyse multi-critère}
Parmi les solutions proposées seulement 8 ont été retenues. La commission souhaite donc établir la meilleure solution parmi les 8 selon 4 critères ( bénéfice, gestion de stock, équilibre commercial et utilisation des machines 2 et 7).\\

Pour cela nous allons effectuer une analyse multi-critère avec la méthode ElectreI car nous ne souhaitons pas classer les actions les unes par rapport aux autres mais juste identifier la meilleure. Pour cela nous allons utiliser le tableau des jugements fourni. Dans un premier temps la méthode sera appliquée au tableau brut, puis au tableau auquel on aura affecté un coefficient par critère selon l'importance et enfin au tableau avec les coefficients et une mise à l'échelle des valeurs en fonction du coefficient.\\ 

\subsection{Analyse multi-critère sans pondération}
A partir du tableau des jugements nous allons construire la matrice de concordance qui va nous permettre de trouver les actions dominantes et celles non dominantes à éliminer. Puis nous allons déterminer la matrice de discordance pour ensuite déduire le tableau de surclassement et son graphique qui, en fonction des seuils choisis pour la concordance et la discordance va donner les ou la meilleure solution.\\

Tableau\\
Concordance, Discordance\\ explication méthode et résultats
Surclassement, Graphique\\ explication méthode et résultats

Cette analyse nous a permis de déterminer que les actions 2,3,5 sont les meilleures faces à l'ensemble des 4 critères.

\subsection{Analyse multi-critère avec pondération}
Nous allons maintenant ré-faire l'analyse en affectant un coefficient à chaque critère selon l'importance attribuée. Nous avons choisi d'accorder le coefficient le plus important à la gestion du stock car c'est le critère qui avait émergé lors de l'analyse précédente (partie 2), puis le bénéfice est aussi un des axes principaux de la stratégie et enfin les deux derniers critères sont secondaires pour la réussite de l'entreprise bien que toujours importants pour sa stabilité dans le temps.
L'application de la méthode reste donc presque la même mais les résultats changent.\\

Tableau\\
Concordance, Discordance\\ explication méthode et résultats
Surclassement, Graphique\\ explication méthode et résultats

Cette analyse nous a permis de déterminer que l'action 3 est la meilleure faces à l'ensemble des 4 critères pondérés par leur importance.

\subsection{Analyse multi-critère avec pondération et mise à l'échelle}
Nous allons maintenant mettre à l'échelle les valeurs du tableau des jugements selon les intervalles et la formule qui suivent : 
Trucs du cours ici.

L'application de la méthode reste donc presque la même mais les résultats changent.\\

Tableau\\
Concordance, Discordance\\ explication méthode et résultats
Surclassement, Graphique\\ explication méthode et résultats

Cette analyse nous a permis de déterminer que les action 3 et 4 sont les meilleures faces à l'ensemble des 4 critères pondérés par leur importance et à la mise à l'échelle.


\end{document}
\geometry{margin=2cm}
